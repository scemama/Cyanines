\documentclass[aip,jcp,reprint,showkeys]{revtex4-1}
\usepackage{graphicx,dcolumn,bm,xcolor,microtype,hyperref,multirow,amscd,amsmath,amssymb,amsfonts,physics}

\newcommand{\alert}[1]{\textcolor{red}{#1}}
\newcommand{\cdash}{\multicolumn{1}{c}{---}}
\newcommand{\mc}{\multicolumn}
\newcommand{\mcc}[1]{\multicolumn{1}{c}{#1}}
\newcommand{\mr}{\multirow}

% dressing symbol
\newcommand{\DrSym}[1]{\Tilde{#1}}


% wave functions and orbitals
\newcommand{\PsiT}{\Psi_\text{T}}
\newcommand{\ExactPsi}{\Phi}
\newcommand{\Js}{J}
\newcommand{\sdet}{D}
\newcommand{\MO}[1]{\phi_{#1}}
\newcommand{\GTO}[1]{\chi_{#1}}

% coefficients
\newcommand{\cCI}[1]{c_{#1}}
\newcommand{\cGTO}[2]{c_{#1 #2}}

% hat operators 
\newcommand{\hH}{\Hat{H}}
\newcommand{\hI}{\Hat{I}}
\newcommand{\hP}{\Hat{P}}
\newcommand{\hF}{\Hat{F}}

% matrices
\newcommand{\bF}{\bm{F}}
\newcommand{\bH}{\bm{H}}
\newcommand{\bP}{\bm{P}}
\newcommand{\eigval}{\bm{\varepsilon}}


% matrix elements
\newcommand{\HEl}[2]{H_{#1 #2}}
\newcommand{\FEl}[2]{F_{#1 #2}}
\newcommand{\MOeigval}[1]{\varepsilon_{#1}}

% bold symbols
\newcommand{\br}{\bm{r}}
\newcommand{\bR}{\bm{R}}
\newcommand{\bA}{\mathbf{A}}
\newcommand{\bB}{\mathbf{B}}
\newcommand{\bC}{\mathbf{C}}
\newcommand{\bD}{\mathbf{D}}
\newcommand{\bO}{\mathbf{0}}
\newcommand{\ba}{\bm{a}}
\newcommand{\occ}{\text{occ}}
\newcommand{\virt}{\text{virt}}

% energies
\newcommand{\EHF}{E_\text{HF}}
\newcommand{\Ec}{E_\text{c}}
\newcommand{\EFCI}{E_\text{FCI}}
\newcommand{\EDMC}{E_\text{DMC}}

% methods
\newcommand{\SCSC}{(SC)$^2$}

\begin{document}	

\title{Electronic excitations in cyanines with state-of-the-art multireference methods}	

\author{Yann Garniron}
\affiliation{Laboratoire de Chimie et Physique Quantiques, Universit\'e de Toulouse, CNRS, UPS, France}
\author{Emmanuel Giner}
\affiliation{Max Planck Institue for Solid State Research, Heisenbergstra{\ss}e 1, 70569, Germany}
\author{Michel Caffarel}
\affiliation{Laboratoire de Chimie et Physique Quantiques, Universit\'e de Toulouse, CNRS, UPS, France}
\author{Jean-Paul Malrieu}
\affiliation{Laboratoire de Chimie et Physique Quantiques, Universit\'e de Toulouse, CNRS, UPS, France}
\author{Anthony Scemama}
\email[Corresponding author: ]{scemama@irsamc.ups-tlse.fr}
\affiliation{Laboratoire de Chimie et Physique Quantiques, Universit\'e de Toulouse, CNRS, UPS, France}
\author{Pierre-Fran{\c c}ois Loos}
\email[Corresponding author: ]{loos@irsamc.ups-tlse.fr}
\affiliation{Laboratoire de Chimie et Physique Quantiques, Universit\'e de Toulouse, CNRS, UPS, France}

\begin{abstract}
This is the abstract
\end{abstract}

\keywords{cyanine; excited states; multireference methods}

\maketitle

%----------------------------------------------------------------
\section{Introduction}
%----------------------------------------------------------------
Unless otherwise stated, atomic units are used throughout.

%----------------------------------------------------------------
\section{Computational details}
%----------------------------------------------------------------

%----------------------------------------------------------------
\section{Conclusion}
%----------------------------------------------------------------


%----------------------------------------------------------------
\begin{acknowledgments}
This work was performed using HPC resources from CALMIP (Toulouse) under allocation 2016-0510 and from GENCI-TGCC (Grant 2016-08s015).
\end{acknowledgments}
%----------------------------------------------------------------

\bibliography{cyanine}

%%% TABLE 1 %%%
\begin{table*}
	\caption{
	\label{tab:GS-energy}
	Ground-state energy of cyanines for various methods and basis sets.}
	\begin{ruledtabular}
	\begin{tabular}{llccccc}
		Method		&	Basis		&	\mc{5}{c}{Ground-state energy of cyanines}				\\
										\cline{3-7}
					&				&	C3		&	C5		&	C7		&	C9		&	C11		\\
					\hline
		HF			&	AVDZ		&			&			&			&			&			\\
					&	AVTZ		&			&			&			&			&			\\
					&	AVQZ		&			&			&			&			&			\\
					\hline
		CISD		&	AVDZ		&			&			&			&			&			\\
					&	AVTZ		&			&			&			&			&			\\
					&	AVQZ		&			&			&			&			&			\\
					\hline
		CAS			&	AVDZ		&			&			&			&			&			\\
					&	AVTZ		&			&			&			&			&			\\
					&	AVQZ		&			&			&			&			&			\\
					\hline
		\SCSC-CI	&	AVDZ		&			&			&			&			&			\\
					&	AVTZ		&			&			&			&			&			\\
					&	AVQZ		&			&			&			&			&			\\
					\hline
		DD-CI		&	AVDZ		&			&			&			&			&			\\
					&	AVTZ		&			&			&			&			&			\\
					&	AVQZ		&			&			&			&			&			\\
					\hline
		MR-CISD		&	AVDZ		&			&			&			&			&			\\
					&	AVTZ		&			&			&			&			&			\\
					&	AVQZ		&			&			&			&			&			\\
					\hline
		DD-CC		&	AVDZ		&			&			&			&			&			\\
					&	AVTZ		&			&			&			&			&			\\
					&	AVQZ		&			&			&			&			&			\\
					\hline
		MR-CCSD		&	AVDZ		&			&			&			&			&			\\
					&	AVTZ		&			&			&			&			&			\\
					&	AVQZ		&			&			&			&			&			\\
					\hline
		MR-CCSD(T)	&	AVDZ		&			&			&			&			&			\\
					&	AVTZ		&			&			&			&			&			\\
					&	AVQZ		&			&			&			&			&			\\
					\hline
		FCI/CIPSI	&	AVDZ		&			&			&			&			&			\\
					&	AVTZ		&			&			&			&			&			\\
					&	AVQZ		&			&			&			&			&			\\
	\end{tabular}		
	\end{ruledtabular}		
\end{table*}		
%%%  %%%

%%% TABLE 2 %%%
\begin{table*}
	\caption{
	\label{tab:ES-energy}
	Excited-state energy of cyanines for various methods and basis sets.}
	\begin{ruledtabular}
	\begin{tabular}{llccccc}
		Method		&	Basis		&	\mc{5}{c}{Excited-state energy of cyanines}				\\
										\cline{3-7}
					&				&	C3		&	C5		&	C7		&	C9		&	C11		\\
					\hline
		HF			&	AVDZ		&			&			&			&			&			\\
					&	AVTZ		&			&			&			&			&			\\
					&	AVQZ		&			&			&			&			&			\\
					\hline
		CISD		&	AVDZ		&			&			&			&			&			\\
					&	AVTZ		&			&			&			&			&			\\
					&	AVQZ		&			&			&			&			&			\\
					\hline
		CAS			&	AVDZ		&			&			&			&			&			\\
					&	AVTZ		&			&			&			&			&			\\
					&	AVQZ		&			&			&			&			&			\\
					\hline
		\SCSC-CI	&	AVDZ		&			&			&			&			&			\\
					&	AVTZ		&			&			&			&			&			\\
					&	AVQZ		&			&			&			&			&			\\
					\hline
		DD-CI		&	AVDZ		&			&			&			&			&			\\
					&	AVTZ		&			&			&			&			&			\\
					&	AVQZ		&			&			&			&			&			\\
					\hline
		MR-CISD		&	AVDZ		&			&			&			&			&			\\
					&	AVTZ		&			&			&			&			&			\\
					&	AVQZ		&			&			&			&			&			\\
					\hline
		DD-CC		&	AVDZ		&			&			&			&			&			\\
					&	AVTZ		&			&			&			&			&			\\
					&	AVQZ		&			&			&			&			&			\\
					\hline
		MR-CCSD		&	AVDZ		&			&			&			&			&			\\
					&	AVTZ		&			&			&			&			&			\\
					&	AVQZ		&			&			&			&			&			\\
					\hline
		MR-CCSD(T)	&	AVDZ		&			&			&			&			&			\\
					&	AVTZ		&			&			&			&			&			\\
					&	AVQZ		&			&			&			&			&			\\
					\hline
		FCI/CIPSI	&	AVDZ		&			&			&			&			&			\\
					&	AVTZ		&			&			&			&			&			\\
					&	AVQZ		&			&			&			&			&			\\
	\end{tabular}		
	\end{ruledtabular}		
\end{table*}		
%%%  %%%

%%% TABLE 3 %%%
\begin{table*}
	\caption{
	\label{tab:VE}
	Vertical excitation energy of cyanines (in eV) for various methods and basis sets.}
	\begin{ruledtabular}
	\begin{tabular}{llccccc}
		Method		&	Basis		&	\mc{5}{c}{Vertical excitation energy of cyanines (in eV)}	\\
										\cline{3-7}
					&				&	C3		&	C5		&	C7		&	C9		&	C11		\\
					\hline
		HF			&	AVDZ		&			&			&			&			&			\\
					&	AVTZ		&			&			&			&			&			\\
					&	AVQZ		&			&			&			&			&			\\
					\hline
		CISD		&	AVDZ		&			&			&			&			&			\\
					&	AVTZ		&			&			&			&			&			\\
					&	AVQZ		&			&			&			&			&			\\
					\hline
		CAS			&	AVDZ		&			&			&			&			&			\\
					&	AVTZ		&			&			&			&			&			\\
					&	AVQZ		&			&			&			&			&			\\
					\hline
		\SCSC-CI	&	AVDZ		&			&			&			&			&			\\
					&	AVTZ		&			&			&			&			&			\\
					&	AVQZ		&			&			&			&			&			\\
					\hline
		DD-CI		&	AVDZ		&			&			&			&			&			\\
					&	AVTZ		&			&			&			&			&			\\
					&	AVQZ		&			&			&			&			&			\\
					\hline
		MR-CISD		&	AVDZ		&			&			&			&			&			\\
					&	AVTZ		&			&			&			&			&			\\
					&	AVQZ		&			&			&			&			&			\\
					\hline
		DD-CC		&	AVDZ		&			&			&			&			&			\\
					&	AVTZ		&			&			&			&			&			\\
					&	AVQZ		&			&			&			&			&			\\
					\hline
		MR-CCSD		&	AVDZ		&			&			&			&			&			\\
					&	AVTZ		&			&			&			&			&			\\
					&	AVQZ		&			&			&			&			&			\\
					\hline
		MR-CCSD(T)	&	AVDZ		&			&			&			&			&			\\
					&	AVTZ		&			&			&			&			&			\\
					&	AVQZ		&			&			&			&			&			\\
					\hline
		FCI/CIPSI	&	AVDZ		&			&			&			&			&			\\
					&	AVTZ		&			&			&			&			&			\\
					&	AVQZ		&			&			&			&			&			\\
	\end{tabular}		
	\end{ruledtabular}		
\end{table*}		
%%%  %%%

%%% TABLE 4 %%%
\begin{table*}
	\caption{
	\label{tab:DMC-GS-energy}
	Ground-state DMC energy of cyanines for various trial wave functions.	
	The statistical error is reported in parenthesis.}
	\begin{ruledtabular}
	\begin{tabular}{llccccc}
		Trial wave 	&	Basis		&	\mc{5}{c}{Ground-state DMC energy of cyanines}				\\
										\cline{3-7}
		function	&				&	C3		&	C5		&	C7		&	C9		&	C11		\\
					\hline
		HF			&	AVDZ		&			&			&			&			&			\\
					&	AVTZ		&			&			&			&			&			\\
					&	AVQZ		&			&			&			&			&			\\
					\hline
		CISD		&	AVDZ		&			&			&			&			&			\\
					&	AVTZ		&			&			&			&			&			\\
					&	AVQZ		&			&			&			&			&			\\
					\hline
		CAS			&	AVDZ		&			&			&			&			&			\\
					&	AVTZ		&			&			&			&			&			\\
					&	AVQZ		&			&			&			&			&			\\
					\hline
		\SCSC-CI	&	AVDZ		&			&			&			&			&			\\
					&	AVTZ		&			&			&			&			&			\\
					&	AVQZ		&			&			&			&			&			\\
					\hline
		DD-CI		&	AVDZ		&			&			&			&			&			\\
					&	AVTZ		&			&			&			&			&			\\
					&	AVQZ		&			&			&			&			&			\\
					\hline
		MR-CISD		&	AVDZ		&			&			&			&			&			\\
					&	AVTZ		&			&			&			&			&			\\
					&	AVQZ		&			&			&			&			&			\\
					\hline
		DD-CC		&	AVDZ		&			&			&			&			&			\\
					&	AVTZ		&			&			&			&			&			\\
					&	AVQZ		&			&			&			&			&			\\
					\hline
		MR-CCSD		&	AVDZ		&			&			&			&			&			\\
					&	AVTZ		&			&			&			&			&			\\
					&	AVQZ		&			&			&			&			&			\\
					\hline
		MR-CCSD(T)	&	AVDZ		&			&			&			&			&			\\
					&	AVTZ		&			&			&			&			&			\\
					&	AVQZ		&			&			&			&			&			\\
					\hline
		FCI/CIPSI	&	AVDZ		&			&			&			&			&			\\
					&	AVTZ		&			&			&			&			&			\\
					&	AVQZ		&			&			&			&			&			\\
	\end{tabular}		
	\end{ruledtabular}		
\end{table*}		
%%%  %%%

%%% TABLE 5 %%%
\begin{table*}
	\caption{
	\label{tab:DMC-ES-energy}
	Excited-state DMC energy of cyanines for various trial wave functions.
	The statistical error is reported in parenthesis.}
	\begin{ruledtabular}
	\begin{tabular}{llccccc}
		Trial wave	&	Basis		&	\mc{5}{c}{Excited-state DMC energy of cyanines}				\\
										\cline{3-7}
		function	&				&	C3		&	C5		&	C7		&	C9		&	C11		\\
					\hline
		HF			&	AVDZ		&			&			&			&			&			\\
					&	AVTZ		&			&			&			&			&			\\
					&	AVQZ		&			&			&			&			&			\\
					\hline
		CISD		&	AVDZ		&			&			&			&			&			\\
					&	AVTZ		&			&			&			&			&			\\
					&	AVQZ		&			&			&			&			&			\\
					\hline
		CAS			&	AVDZ		&			&			&			&			&			\\
					&	AVTZ		&			&			&			&			&			\\
					&	AVQZ		&			&			&			&			&			\\
					\hline
		\SCSC-CI	&	AVDZ		&			&			&			&			&			\\
					&	AVTZ		&			&			&			&			&			\\
					&	AVQZ		&			&			&			&			&			\\
					\hline
		DD-CI		&	AVDZ		&			&			&			&			&			\\
					&	AVTZ		&			&			&			&			&			\\
					&	AVQZ		&			&			&			&			&			\\
					\hline
		MR-CISD		&	AVDZ		&			&			&			&			&			\\
					&	AVTZ		&			&			&			&			&			\\
					&	AVQZ		&			&			&			&			&			\\
					\hline
		DD-CC		&	AVDZ		&			&			&			&			&			\\
					&	AVTZ		&			&			&			&			&			\\
					&	AVQZ		&			&			&			&			&			\\
					\hline
		MR-CCSD		&	AVDZ		&			&			&			&			&			\\
					&	AVTZ		&			&			&			&			&			\\
					&	AVQZ		&			&			&			&			&			\\
					\hline
		MR-CCSD(T)	&	AVDZ		&			&			&			&			&			\\
					&	AVTZ		&			&			&			&			&			\\
					&	AVQZ		&			&			&			&			&			\\
					\hline
		FCI/CIPSI	&	AVDZ		&			&			&			&			&			\\
					&	AVTZ		&			&			&			&			&			\\
					&	AVQZ		&			&			&			&			&			\\
	\end{tabular}		
	\end{ruledtabular}		
\end{table*}		
%%%  %%%

%%% TABLE 6 %%%
\begin{table*}
	\caption{
	\label{tab:VE}
	Vertical excitation DMC energy of cyanines (in eV) for various trial wave functions.
	The statistical error is reported in parenthesis.}
	\begin{ruledtabular}
	\begin{tabular}{llccccc}
		Trial wave	&	Basis		&	\mc{5}{c}{Vertical excitation DMC energy of cyanines (in eV)}	\\
										\cline{3-7}
		function	&				&	C3		&	C5		&	C7		&	C9		&	C11		\\
					\hline
		HF			&	AVDZ		&			&			&			&			&			\\
					&	AVTZ		&			&			&			&			&			\\
					&	AVQZ		&			&			&			&			&			\\
					\hline
		CISD		&	AVDZ		&			&			&			&			&			\\
					&	AVTZ		&			&			&			&			&			\\
					&	AVQZ		&			&			&			&			&			\\
					\hline
		CAS			&	AVDZ		&			&			&			&			&			\\
					&	AVTZ		&			&			&			&			&			\\
					&	AVQZ		&			&			&			&			&			\\
					\hline
		\SCSC-CI	&	AVDZ		&			&			&			&			&			\\
					&	AVTZ		&			&			&			&			&			\\
					&	AVQZ		&			&			&			&			&			\\
					\hline
		DD-CI		&	AVDZ		&			&			&			&			&			\\
					&	AVTZ		&			&			&			&			&			\\
					&	AVQZ		&			&			&			&			&			\\
					\hline
		MR-CISD		&	AVDZ		&			&			&			&			&			\\
					&	AVTZ		&			&			&			&			&			\\
					&	AVQZ		&			&			&			&			&			\\
					\hline
		DD-CC		&	AVDZ		&			&			&			&			&			\\
					&	AVTZ		&			&			&			&			&			\\
					&	AVQZ		&			&			&			&			&			\\
					\hline
		MR-CCSD		&	AVDZ		&			&			&			&			&			\\
					&	AVTZ		&			&			&			&			&			\\
					&	AVQZ		&			&			&			&			&			\\
					\hline
		MR-CCSD(T)	&	AVDZ		&			&			&			&			&			\\
					&	AVTZ		&			&			&			&			&			\\
					&	AVQZ		&			&			&			&			&			\\
					\hline
		FCI/CIPSI	&	AVDZ		&			&			&			&			&			\\
					&	AVTZ		&			&			&			&			&			\\
					&	AVQZ		&			&			&			&			&			\\
	\end{tabular}		
	\end{ruledtabular}		
\end{table*}		
%%%  %%%

%%% TABLE 7 %%%
\begin{table*}
	\caption{
	\label{tab:comparison-VE}
	Comparison of vertical excitation energy of cyanines (in eV) for various methods.}
	\begin{ruledtabular}
	\begin{tabular}{lllllllc}
		Method		&	\mc{5}{c}{Vertical excitation energy of cyanines (in eV)}				\\
						\cline{2-6}
						&	C3		&	C5		&	C7		&	C9		&	C11						\\
		\hline
						&	\mc{5}{c}{Wave function methods}							\\
		CISD			&			&			&			&			&			&	This work	\\
		CAS				&			&			&			&			&			&	This work	\\
		\SCSC-CI		&			&			&			&			&			&	This work	\\
		DD-CI			&			&			&			&			&			&	This work	\\
		MR-CISD			&			&			&			&			&			&	This work	\\
		DD-CC			&			&			&			&			&			&	This work	\\
		MR-CCSD			&			&			&			&			&			&	This work	\\
		MR-CCSD(T)		&			&			&			&			&			&	This work	\\
		FCI/CIPSI		&			&			&			&			&			&	This work	\\
		\hline
						&	\mc{5}{c}{DMC calculations}								\\
		CISD			&			&			&			&			&			&	This work	\\
		CAS				&			&			&			&			&			&	This work	\\
		\SCSC-CI		&			&			&			&			&			&	This work	\\
		DD-CI			&			&			&			&			&			&	This work	\\
		MR-CISD			&			&			&			&			&			&	This work	\\
		DD-CC			&			&			&			&			&			&	This work	\\
		MR-CCSD			&			&			&			&			&			&	This work	\\
		MR-CCSD(T)		&			&			&			&			&			&	This work	\\
		FCI/CIPSI		&			&			&			&			&			&	This work	\\
		\hline
						&	\mc{5}{c}{Reference calculations}									\\
		exCC3			&	7.16		&	4.84	&	3.65	&	2.96	&	2.53	&	Ref.~\onlinecite{Send11}	\\
		DMC				&	7.38(2)		&	5.03(2)	&	3.83(2)	&	3.09(2)	&	2.62(2)	&	Ref.~\onlinecite{Send11}	\\
		CASPT2			&	7.19		&	4.69	&	3.53	&	2.81	&	2.46	&	Ref.~\onlinecite{Send11}	\\
		GW/BSE			&				&	4.80	&	3.63	&	2.96	&	2.48	&	Ref.~\onlinecite{Boulanger14}	\\
		\hline
						&	\mc{5}{c}{TD-DFT calculations}													\\
		TD-PBE			&	7.40		&	5.22		&	4.11		&	3.44		&	2.98		&	Ref.~\onlinecite{Send11}	\\
		TD-PBE0			&	7.62		&	5.33		&	4.18		&	3.50		&	3.03		&	Ref.~\onlinecite{Send11}	\\
		TD-B2PLYP		&	7.30		&	5.05		&	3.92		&	3.25		&	2.80		&	Ref.~\onlinecite{Send11}	\\
		TD-CAM-B3LYP	&	7.55		&	5.26		&	4.12		&	3.44		&	2.97		&	Ref.~\onlinecite{Send11}	\\
		TD-MO6-2X		&				&	5.23		&	4.09		&	3.41		&	2.95		&	Ref.~\onlinecite{Jacquemin12}	\\
	\end{tabular}		
	\end{ruledtabular}		
\end{table*}		
%%%  %%%


\end{document}
